% !TeX encoding = usascii
%%%%%%%%%%%%%%%%%%%%%%%%%%%%%%%%%%%%%%%%%%%%%%%%%%%%%%%%%%%%%%%%%%%%%%
%% $Id$
%%%%%%%%%%%%%%%%%%%%%%%%%%%%%%%%%%%%%%%%%%%%%%%%%%%%%%%%%%%%%%%%%%%%%%

%%%%%%%%%%%%%%%%%%%%%%%%%%%%%%%%%%%%%%%%%%%%%%%%%%%%%%%%%%%%%%%%%%%%%%
%%
%% IVT DISSERATION LAYOUT
%% Date: 2007-06-28
%% author:
%%   Michael Balmer, balmer@ivt.baug.ethz.ch
%%
%%%%%%%%%%%%%%%%%%%%%%%%%%%%%%%%%%%%%%%%%%%%%%%%%%%%%%%%%%%%%%%%%%%%%%

%%%%%%%%%%%%%%%%%%%%%%%%%%%%%%%%%%%%%%%%%%%%%%%%%%%%%%%%%%%%%%%%%%%%%%
%%
%% The commands in here HAVE to be defined by the main file and
%% therefore can be used in the layout (commands defined by the
%% main file always starts with '\my')
%%   \mypath := relative path to the paper directory
%%   \myfirstlang := main langugage (german|english)
%%   \mytitlefigure := relative location of the title figure
%%   \mytitle := the title of the paper
%%   \myinstitutionEN := the inst. (english) for which the paper is written
%%   \myinstitutionDE := the inst. (german) for which the paper is written
%%   \mynumber := the number of the paper
%%   \myyear := the year of publishing
%%   \mymonth := the month of publishing
%%   \myday := the day of publishing
%%   \mywordcount := number of words of the paper
%%   \my[first|second|third|fourth|fifth|sisth]author := Author names
%%   \my[first|second|third|fourth|fifth|sisth]authorREF := Author names
%%     for the reference
%%   \my[first|second|third|fourth|fifth|sisth]address := authors affiliation
%%   \myfirstauthortitle := title of the author (typically MSc or similar)
%%   \myfirstauthordoctortitle := title to receive for the dissertation
%%   \myfirstauthorbirthday := date of birth of the author
%%   \myfirstauthorcity := home-city ([b\"urgerort, Kanton] for citizens of
%%     Switzerland or [Country] for a non-Swiss author)
%%   \mykeywordsEN := english keywords
%%   \mykeywordsDE := german keywords
%%
%%%%%%%%%%%%%%%%%%%%%%%%%%%%%%%%%%%%%%%%%%%%%%%%%%%%%%%%%%%%%%%%%%%%%%

%%%%%%%%%%%%%%%%%%%%%%%%%%%%%%%%%%%%%%%%%%%%%%%%%%%%%%%%%%%%%%%%%%%%%%
%%%%%%%%%%%%%%%%%%%%%%%%%%%%%%%%%%%%%%%%%%%%%%%%%%%%%%%%%%%%%%%%%%%%%%
%%
%% Standard latex layout configurations
%%   Here, commands and settings are use which are available
%%   by the latex packages
%%
%%%%%%%%%%%%%%%%%%%%%%%%%%%%%%%%%%%%%%%%%%%%%%%%%%%%%%%%%%%%%%%%%%%%%%
%%%%%%%%%%%%%%%%%%%%%%%%%%%%%%%%%%%%%%%%%%%%%%%%%%%%%%%%%%%%%%%%%%%%%%
%% Type of document:
%% - Paperformat: letterpaper, a4paper, a5paper, b5paper,
%%   executivepaper, legalpaper
%% - Main font size: 10pt, 11pt, 12pt
%% - Formulae setting: - (centred), fleqn (left-aligned)
%% - Numbering of formulae: - (right-aligned), leqno (left-aligned)
%% - New page after title: titlepage, notitlepage
%% - Number of columns per page: onecolumn, twocolumn
%% - Page style: oneside, twoside
%% - Paper rotation: - (protrait), landscape
%% - Chapter start: openright, openany
%% - Mark overfull boxes: draft, final
\documentclass[a4paper,12pt,fleqn,titlepage,onecolumn,twoside,openany,final]{report}
%%%%%%%%%%%%%%%%%%%%%%%%%%%%%%%%%%%%%%%%%%%%%%%%%%%%%%%%%%%%%%%%%%%%%%
%% borders, margrins and offset
\usepackage[a4paper,left=1.0in,right=1.0in,top=0.5in,bottom=0.5in,includeheadfoot]{geometry}
%%%%%%%%%%%%%%%%%%%%%%%%%%%%%%%%%%%%%%%%%%%%%%%%%%%%%%%%%%%%%%%%%%%%%%
%% providing if-then-else command:
\usepackage{ifthen}
%%%%%%%%%%%%%%%%%%%%%%%%%%%%%%%%%%%%%%%%%%%%%%%%%%%%%%%%%%%%%%%%%%%%%%
%% default language:
\ifthenelse{\equal{\myfirstlang}{german}}{%
  \usepackage[english,german]{babel}%
}{%
  \usepackage[german,english]{babel}%
}
%%%%%%%%%%%%%%%%%%%%%%%%%%%%%%%%%%%%%%%%%%%%%%%%%%%%%%%%%%%%%%%%%%%%%%
%% Header and footer definition:
\usepackage{fancyhdr}%
\pagestyle{fancy}%
\fancyhf{}%
\fancyhead[R]{\slshape \footnotesize \nouppercase{\myyear}}%
\fancyhead[L]{\slshape \footnotesize \nouppercase{\mytitle}}%
\fancyfoot[C]{\footnotesize \thepage}%
\renewcommand{\headrulewidth}{0.5pt}%
\renewcommand{\footrulewidth}{0pt}%
%%%%%%%%%%%%%%%%%%%%%%%%%%%%%%%%%%%%%%%%%%%%%%%%%%%%%%%%%%%%%%%%%%%%%%
%% paragraph settings:
\setlength{\parindent}{0in}%
\setlength{\parskip}{10pt}
%%%%%%%%%%%%%%%%%%%%%%%%%%%%%%%%%%%%%%%%%%%%%%%%%%%%%%%%%%%%%%%%%%%%%%
%% caption settings:
\usepackage[nooneline,format=hang]{caption}
%%%%%%%%%%%%%%%%%%%%%%%%%%%%%%%%%%%%%%%%%%%%%%%%%%%%%%%%%%%%%%%%%%%%%%
%% Define the depth of numbering parts,chapter,sections and paragraphs:
%%   Numbers representing the depth of sectional units:
%%   -1 = \part    (in book or report document classes)
%%    0 = \chapter (in book or report document classes)
%%    0 = \part    (in article document classes)
%%    1 = \section
%%    2 = \subsection
%%    3 = \subsubsection
%%    4 = \paragraph
%%    5 = \subparagraph
\setcounter{secnumdepth}{3}
%%%%%%%%%%%%%%%%%%%%%%%%%%%%%%%%%%%%%%%%%%%%%%%%%%%%%%%%%%%%%%%%%%%%%%
%% citation style:
\usepackage[round]{natbib}
\ifthenelse{\equal{\myfirstlang}{german}}{%
  \bibliographystyle{\mypath../_latexfiles/styles/template_ivt-ger}%
}{%
  \bibliographystyle{\mypath../_latexfiles/styles/template_ivt-eng}%
}
%%%%%%%%%%%%%%%%%%%%%%%%%%%%%%%%%%%%%%%%%%%%%%%%%%%%%%%%%%%%%%%%%%%%%%
%% Font:
\usepackage{times}
%%%%%%%%%%%%%%%%%%%%%%%%%%%%%%%%%%%%%%%%%%%%%%%%%%%%%%%%%%%%%%%%%%%%%%
%% To prevent overfull boxes
%%   it is quite nice, but in special cases, you will have too large
%%   speaces between a two words of the same line.
\sloppy
%%%%%%%%%%%%%%%%%%%%%%%%%%%%%%%%%%%%%%%%%%%%%%%%%%%%%%%%%%%%%%%%%%%%%%
%% providing umlauts:
\usepackage[latin1]{inputenc}
\usepackage[T1]{fontenc}
%%%%%%%%%%%%%%%%%%%%%%%%%%%%%%%%%%%%%%%%%%%%%%%%%%%%%%%%%%%%%%%%%%%%%%
%% use hyper-refs for URL's and citations
\usepackage{hyperref}
%% line breaks for URL's
\usepackage{url}
%%%%%%%%%%%%%%%%%%%%%%%%%%%%%%%%%%%%%%%%%%%%%%%%%%%%%%%%%%%%%%%%%%%%%%
%% line spacing
\usepackage{setspace}
\onehalfspacing
%%%%%%%%%%%%%%%%%%%%%%%%%%%%%%%%%%%%%%%%%%%%%%%%%%%%%%%%%%%%%%%%%%%%%%
%% letter spacing
\usepackage{soul}
%%%%%%%%%%%%%%%%%%%%%%%%%%%%%%%%%%%%%%%%%%%%%%%%%%%%%%%%%%%%%%%%%%%%%%
%% no indentation for formulas:
\usepackage[fleqn]{amsmath}
\setlength\mathindent{0pt}
%%%%%%%%%%%%%%%%%%%%%%%%%%%%%%%%%%%%%%%%%%%%%%%%%%%%%%%%%%%%%%%%%%%%%%
%% providing graphics:
\usepackage{graphics}
\usepackage{graphicx}
%%%%%%%%%%%%%%%%%%%%%%%%%%%%%%%%%%%%%%%%%%%%%%%%%%%%%%%%%%%%%%%%%%%%%%
%% sideways figures and tables:
\usepackage{rotating}
%%%%%%%%%%%%%%%%%%%%%%%%%%%%%%%%%%%%%%%%%%%%%%%%%%%%%%%%%%%%%%%%%%%%%%
%% sub-figures:
\usepackage[FIGTOPCAP]{subfigure}
\def\subfigtopskip{0pt}
\def\subfigbottomskip{5pt}
\def\subfigcapskip{0pt}
%%%%%%%%%%%%%%%%%%%%%%%%%%%%%%%%%%%%%%%%%%%%%%%%%%%%%%%%%%%%%%%%%%%%%%
%% figures:
%%   The following are sometimes needed to avoid pushing
%%   the figs to the end of the text.
\def\textfraction{0.0}
\def\topfraction{0.9999}
\def\floatpagefraction{0.9}
%%%%%%%%%%%%%%%%%%%%%%%%%%%%%%%%%%%%%%%%%%%%%%%%%%%%%%%%%%%%%%%%%%%%%%
%% tables:
\usepackage{tabularx}
\usepackage{multirow}
%%%%%%%%%%%%%%%%%%%%%%%%%%%%%%%%%%%%%%%%%%%%%%%%%%%%%%%%%%%%%%%%%%%%%%
%% pretty printing:
\usepackage{listings}
%%%%%%%%%%%%%%%%%%%%%%%%%%%%%%%%%%%%%%%%%%%%%%%%%%%%%%%%%%%%%%%%%%%%%%
%% XML code setup:
\lstloadlanguages{XML}
%%
\lstset {
  showstringspaces=false,
  basicstyle=\ttfamily\footnotesize,
  lineskip=0pt,
  breaklines=true,
  breakatwhitespace=true,
  breakindent=12pt,
  fontadjust=true,
  keywordstyle=\bfseries,
  commentstyle=\bfseries,
  stringstyle=\bfseries,
  xleftmargin=0mm,
  xrightmargin=0mm,
  tabsize=2
}
%%%%%%%%%%%%%%%%%%%%%%%%%%%%%%%%%%%%%%%%%%%%%%%%%%%%%%%%%%%%%%%%%%%%%%
%% convenient referencing:
\usepackage[capitalize]{cleveref}
%%%%%%%%%%%%%%%%%%%%%%%%%%%%%%%%%%%%%%%%%%%%%%%%%%%%%%%%%%%%%%%%%%%%%%

%%%%%%%%%%%%%%%%%%%%%%%%%%%%%%%%%%%%%%%%%%%%%%%%%%%%%%%%%%%%%%%%%%%%%%

%%%%%%%%%%%%%%%%%%%%%%%%%%%%%%%%%%%%%%%%%%%%%%%%%%%%%%%%%%%%%%%%%%%%%%
%%%%%%%%%%%%%%%%%%%%%%%%%%%%%%%%%%%%%%%%%%%%%%%%%%%%%%%%%%%%%%%%%%%%%%
%%
%% The following defines language specific words
%%   These are internal commands. They are not used in the main file.
%%   Langugage specific word commands always starts with '\word'
%%
%%%%%%%%%%%%%%%%%%%%%%%%%%%%%%%%%%%%%%%%%%%%%%%%%%%%%%%%%%%%%%%%%%%%%%
%%%%%%%%%%%%%%%%%%%%%%%%%%%%%%%%%%%%%%%%%%%%%%%%%%%%%%%%%%%%%%%%%%%%%%
%% and/und
\newcommand{\wordand}{\iflanguage{english}{and}{\iflanguage{german}{und}{\langerrmessage}}}
%%%%%%%%%%%%%%%%%%%%%%%%%%%%%%%%%%%%%%%%%%%%%%%%%%%%%%%%%%%%%%%%%%%%%%
%% phone/Tel
\newcommand{\wordphone}{\iflanguage{english}{phone}{\iflanguage{german}{Tel}{\langerrmessage}}}
%%%%%%%%%%%%%%%%%%%%%%%%%%%%%%%%%%%%%%%%%%%%%%%%%%%%%%%%%%%%%%%%%%%%%%
%% fax/Fax
\newcommand{\wordfax}{\iflanguage{english}{fax}{\iflanguage{german}{Fax}{\langerrmessage}}}
%%%%%%%%%%%%%%%%%%%%%%%%%%%%%%%%%%%%%%%%%%%%%%%%%%%%%%%%%%%%%%%%%%%%%%
%% Preferred citation style/Bevorzugter Zitierstil
\newcommand{\wordprefcit}{\iflanguage{english}{Preferred citation style}{\iflanguage{german}{Bevorzugter Zitierstil}{\langerrmessage}}}
%%%%%%%%%%%%%%%%%%%%%%%%%%%%%%%%%%%%%%%%%%%%%%%%%%%%%%%%%%%%%%%%%%%%%%
%% Keywords/Schl\"usselw\"orter
\newcommand{\wordkeywords}{\iflanguage{english}{Keywords}{\iflanguage{german}{Schl\"usselw\"orter}{\langerrmessage}}}
%%%%%%%%%%%%%%%%%%%%%%%%%%%%%%%%%%%%%%%%%%%%%%%%%%%%%%%%%%%%%%%%%%%%%%
%% Quelle/Source
\newcommand{\wordsource}{\iflanguage{english}{Source:\ }{\iflanguage{german}{Quelle:\ }{\langerrmessage}}}
%%%%%%%%%%%%%%%%%%%%%%%%%%%%%%%%%%%%%%%%%%%%%%%%%%%%%%%%%%%%%%%%%%%%%%

%%%%%%%%%%%%%%%%%%%%%%%%%%%%%%%%%%%%%%%%%%%%%%%%%%%%%%%%%%%%%%%%%%%%%%
%%%%%%%%%%%%%%%%%%%%%%%%%%%%%%%%%%%%%%%%%%%%%%%%%%%%%%%%%%%%%%%%%%%%%%
%%
%% The following defines other internal commands
%%   Internal command are not used by the main file.
%%   Internal commands always starts with '\internal'
%%
%%%%%%%%%%%%%%%%%%%%%%%%%%%%%%%%%%%%%%%%%%%%%%%%%%%%%%%%%%%%%%%%%%%%%%
%%%%%%%%%%%%%%%%%%%%%%%%%%%%%%%%%%%%%%%%%%%%%%%%%%%%%%%%%%%%%%%%%%%%%%
%% user defined german keywords/user defined english keywords
%%   The command only returns a string which are defined by the
%%   main file (\mykeywordsEN or \mykeywordsDE)
\newcommand{\internkeywords}{\iflanguage{english}{\mykeywordsEN}{\iflanguage{german}{\mykeywordsDE}{\langerrmessage}}}
%%%%%%%%%%%%%%%%%%%%%%%%%%%%%%%%%%%%%%%%%%%%%%%%%%%%%%%%%%%%%%%%%%%%%%
%% papertype definition
%%   i.e. working paper/Arbeitsberichte Verkehrs- und Raumplanung
%%   i.e. disseration/Doktorarbeit
\newcommand{\internpapertype}{\iflanguage{english}{PhD-Thesis}{\iflanguage{german}{Dissertation}{\langerrmessage}}}
%%%%%%%%%%%%%%%%%%%%%%%%%%%%%%%%%%%%%%%%%%%%%%%%%%%%%%%%%%%%%%%%%%%%%%
%% \internmakethreecolumns{entry1}{entry2}{entry3}
%%   creates a table with three columns containing the given entries
\newcommand{\internmakethreecolumns}[3]{
  \noindent
  \begin{tabular*}{\textwidth}{@{}l@{}l@{}l@{}}
    #1 & #2 & #3 \\
  \end{tabular*}
}
%%%%%%%%%%%%%%%%%%%%%%%%%%%%%%%%%%%%%%%%%%%%%%%%%%%%%%%%%%%%%%%%%%%%%%
%% citation
%%   returns the language specific citation of the paper
\newcommand{\interncitation}{%
  \myfirstauthorREF\ (\myyear) \mytitle, \internpapertype, %
  \iflanguage{english}{ETH-Zurich, Zurich.}%
  {\iflanguage{german}{ETH-Z\"urich, Z\"urich.}%
   {\langerrmessage}}}
%%%%%%%%%%%%%%%%%%%%%%%%%%%%%%%%%%%%%%%%%%%%%%%%%%%%%%%%%%%%%%%%%%%%%%

%%%%%%%%%%%%%%%%%%%%%%%%%%%%%%%%%%%%%%%%%%%%%%%%%%%%%%%%%%%%%%%%%%%%%%
%%%%%%%%%%%%%%%%%%%%%%%%%%%%%%%%%%%%%%%%%%%%%%%%%%%%%%%%%%%%%%%%%%%%%%
%%
%% Standard commands
%%   The following commands >>>HAVE<<< to be defined, because
%%   they are called by the main file.
%%   Standard commands always starts with '\create'. (except
%%   '\switchlanguage' since it would sound stupid ^_^ and
%%   '\ackname' since it is the standard def in babel the package)
%%
%%%%%%%%%%%%%%%%%%%%%%%%%%%%%%%%%%%%%%%%%%%%%%%%%%%%%%%%%%%%%%%%%%%%%%
%% switch language
\newcommand{\switchlanguage}{\iflanguage{english}{\selectlanguage{german}}{\selectlanguage{english}}}
%%%%%%%%%%%%%%%%%%%%%%%%%%%%%%%%%%%%%%%%%%%%%%%%%%%%%%%%%%%%%%%%%%%%%%
%% switch language
\newcommand{\ackname}{\iflanguage{english}{Acknowledgement}{\iflanguage{german}{Danksagung}{\langerrmessage}}}
%%%%%%%%%%%%%%%%%%%%%%%%%%%%%%%%%%%%%%%%%%%%%%%%%%%%%%%%%%%%%%%%%%%%%%
%% Figure definition
%%   \createfigure
%%     {<short caption>}
%%     {<long caption>}
%%     {<\label{label}>}
%%     {<\includegraphics[...]{figure}>}
%%     {<source>or<>}
\newcommand{\createfigure}[5]{
  \begin{figure}
    \caption[#1]{#2} #3
    \vspace{0.5em}
    \hrule
    \begin{center}
      #4\\
    \end{center}
    \ifthenelse
      {\equal{#5}{}}
      {}
      {\wordsource #5 \vspace{0.5em}}
    \hrule
  \end{figure}
}
%%%%%%%%%%%%%%%%%%%%%%%%%%%%%%%%%%%%%%%%%%%%%%%%%%%%%%%%%%%%%%%%%%%%%%
%% Sideways figure definition
%%   \createfigure
%%     {<short caption>}
%%     {<long caption>}
%%     {<\label{label}>}
%%     {<\includegraphics[...]{figure}>}
%%     {<source>or<>}
\newcommand{\createsidewaysfigure}[5]{
  \begin{sidewaysfigure}
    \caption[#1]{#2} #3
    \vspace{0.5em}
    \hrule
    \begin{center}
      #4\\
    \end{center}
    \ifthenelse
      {\equal{#5}{}}
      {}
      {\wordsource #5 \vspace{0.5em}}
    \hrule
  \end{sidewaysfigure}
}
%%%%%%%%%%%%%%%%%%%%%%%%%%%%%%%%%%%%%%%%%%%%%%%%%%%%%%%%%%%%%%%%%%%%%%
%% Sub-figure:
%%   \createsubfigure
%%     {<caption>}
%%     {<\includegraphics[...]{figure}>}
%%     {<\label{label}>}
%%     {<\\>or<>}%
\newcommand{\createsubfigure}[4]{
  \subfigure[#1]{%
    #2%
    #3%
  }#4
}
%%%%%%%%%%%%%%%%%%%%%%%%%%%%%%%%%%%%%%%%%%%%%%%%%%%%%%%%%%%%%%%%%%%%%%
%% Table:
%%   \createtable
%%     {<short caption>}
%%     {<long caption>}
%%     {<\label{label}>}
%%     {<\begin{tabular}...\end{tabular}>}
%%     {<source>or<>}
\newcommand{\createtable}[5]{
  \begin{table}
    \caption[#1]{#2} #3
    \vspace{0.5em}
    \hrule
    \begin{center}
      #4\\
    \end{center}
    \ifthenelse
      {\equal{#5}{}}
      {}
      {\wordsource #5 \vspace{0.5em}}
    \hrule
  \end{table}
}
%%%%%%%%%%%%%%%%%%%%%%%%%%%%%%%%%%%%%%%%%%%%%%%%%%%%%%%%%%%%%%%%%%%%%%
%% Sideways table definition
%%   \createsidewaystable
%%     {<short caption>}
%%     {<long caption>}
%%     {<\label{label}>}
%%     {<\begin{tabular}...\end{tabular}>}
%%     {<source>or<>}
\newcommand{\createsidewaystable}[5]{
  \begin{sidewaystable}
    \caption[#1]{#2} #3
    \vspace{0.5em}
    \hrule
    \begin{center}
      #4\\
    \end{center}
    \ifthenelse
      {\equal{#5}{}}
      {}
      {\wordsource #5 \vspace{0.5em}}
    \hrule
  \end{sidewaystable}
}
%%%%%%%%%%%%%%%%%%%%%%%%%%%%%%%%%%%%%%%%%%%%%%%%%%%%%%%%%%%%%%%%%%%%%%
%% XML-figure
%%   \createxmlfigure
%%     {<short caption>}
%%     {<long caption>}
%%     {<\label{label}>}
%%     {<the/file/with/the/xml/code/to/include>}
%%     {<source>or<>}
\newcommand{\createxmlfigure}[5]{
  \begin{figure}
    \caption[#1]{#2} #3
    \vspace{0.5em}
    \hrule
    \lstset{language=XML}
    \lstinputlisting{#4}
    \ifthenelse
      {\equal{#5}{}}
      {}
      {\wordsource #5 \vspace{0.5em}}
    \hrule
  \end{figure}
}
%%%%%%%%%%%%%%%%%%%%%%%%%%%%%%%%%%%%%%%%%%%%%%%%%%%%%%%%%%%%%%%%%%%%%%
%% Contact
%%   \createcontact
%%     {<Name>}
%%     {<andreas line 1>}
%%     {<andreas line 1>}
%%     {<andreas line 3>}
%%     {<phone number>}
%%     {<fax number>}
%%     {<email address>}
\newcommand{\createcontact}[7]{
  \ifthenelse{\equal{#1}{}}{%
    \noindent\parbox[][][l]{0.33\textwidth}{
    }%
  }{%
    \noindent\parbox[][][l]{0.33\textwidth}{
      #1\newline
      #2\newline
      #3\newline
      #4\newline
      \wordphone: #5\newline
      \wordfax: #6\newline
      #7\newline
    }%
  }%
}
%%%%%%%%%%%%%%%%%%%%%%%%%%%%%%%%%%%%%%%%%%%%%%%%%%%%%%%%%%%%%%%%%%%%%%
%% Titlepage definition
\newcommand{\createtitlepage}{
  \begin{titlepage}
    \hrule
    \noindent\parbox[][9cm][c]{\textwidth}{
      \ifthenelse
        {\equal{\mytitlefigure}{}}
        {}
        {\begin{center}\includegraphics[width=\textwidth,totalheight=9cm,keepaspectratio=true]{\mytitlefigure}\end{center}}
    }
    \vspace{0.1in}

    \hrule

    \vspace{0.2in}

    \noindent\LARGE\textbf{\mytitle}

    \vspace{0.3in}

    \noindent\parbox[][3cm][l]{\textwidth}{
      \large\textbf{\myfirstauthor}
    }

    \vfill

    \noindent
    \begin{singlespace}
    \begin{tabular*}{\textwidth}{@{}l@{\extracolsep{\fill}}r@{}}
      \large\textbf{\internpapertype}   & \large\textbf{\mynumber} \\
                                        & \large\textbf{\myyear}   \\
    \end{tabular*}
    \end{singlespace}

    \vspace{0.25in}

    \noindent
    \begin{tabular*}{\textwidth}{@{}l@{\extracolsep{\fill}}r@{}}
      \includegraphics[width=2.5in]{\mypath../_latexfiles/logos/ivt-logo} & \includegraphics[width=2.5in]{\mypath../_latexfiles/logos/eth-logo} \\
    \end{tabular*}
    \vfill
  \end{titlepage}
}
%%%%%%%%%%%%%%%%%%%%%%%%%%%%%%%%%%%%%%%%%%%%%%%%%%%%%%%%%%%%%%%%%%%%%%
%% Abstract definition
\newcommand{\createabstract}[1]{
  \noindent
  \begin{tabular*}{\textwidth}{@{}l@{\extracolsep{\fill}}r@{}}
    \internpapertype & \mynumber \\
  \end{tabular*}

  \vspace{0.25in}

  \noindent\textbf{\Large\mytitle}

  \begin{singlespace}
  \ifthenelse{\equal{\myfourthauthor}{}}{%
    \internmakethreecolumns{\myfirstaddress}{\mysecondaddress}{\mythirdaddress}%
  }{%
    \internmakethreecolumns{\myfirstaddress}{\mysecondaddress}{\mythirdaddress}%
    \internmakethreecolumns{\myfourthaddress}{\myfifthaddress}{\mysixthaddress}%
  }%
  \end{singlespace}

  \noindent\myyear

  \vspace{0.25in} \noindent \textbf{\sffamily\Large \abstractname}

  \begin{singlespace}
  #1
  \end{singlespace}

  \vfill

  \parbox[bottom][][b]{\textwidth}{%
    \vspace{0.25in} \noindent \textbf{\sffamily\Large \wordkeywords}\\
    \internkeywords

    \vspace{0.25in} \noindent \textbf{\sffamily\Large \wordprefcit}\\
    \interncitation
  }

  \eject
}
%%%%%%%%%%%%%%%%%%%%%%%%%%%%%%%%%%%%%%%%%%%%%%%%%%%%%%%%%%%%%%%%%%%%%%
