%%%%%%%%%%%%%%%%%%%%%%%%%%%%%%%%%%%%%%%%%%%%%%%%%%%%%%%%%%%%%%%%%%%%%%
%
\section{Summary and Important Notes}
%
%%%%%%%%%%%%%%%%%%%%%%%%%%%%%%%%%%%%%%%%%%%%%%%%%%%%%%%%%%%%%%%%%%%%%%

This example paper gives you a short overview of how to write a paper
in \LaTeX{} inside the IVT \LaTeX{}-Bib\TeX{} environment. 
Some of the
concepts are general while other are made for the use at the IVT using
the IVT \LaTeX{} environment.

If you want to know more about \LaTeX, there is a pretty good and not
too long introduction under
\url{http://www.ctan.org/tex-archive/info/lshort/english/lshort.pdf}.
Even more can be found under \url{http://www.ctan.org/}. Also using
Google search engine helps a lot.

At last, there are some very important notes about writing in \LaTeX:
Since it is not a WYSIWYG way of writing papers, you always need to
``compile'' it to pdf. Unfortunately, if you do something wrong (i.e.\
forgetting a closing bracket, using non ISO characters, or using
special character that \TeX\ will interpret in another way) then error
messages will appear that are very very hard to understand. Therefore,
if you add something special, i.e.\ a figure, a table, a new Bib\TeX\
entry or a formula, compile the paper right before and right after you
added such stuff. If errors occurs, then you will at least know that
the produced error is caused by the last thing you did. That helps a
lot!

And something helps a lot, too: There are already many papers,
dissertations, CVs, misc stuff, etc., that you can find under
\texttt{ivt/doc}, that shows you very good examples what you can do
with \LaTeX.

If you have questions do not hesitate to ask someone at the IVT with a
bit of experience (Kirill, Basil, \ldots).


